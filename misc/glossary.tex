\newglossaryentry{imscp}
  {
	name=Paquetage de Contenu IMS,
	symbol=Paquetage de Contenu IMS,
	description=
	{
	  (ou IMS Content Packaging) est un format de paquet de contenu créé par
	  l'IMS Global Learning Consortium. C'est un format standardisé qui permet
	  l'échange -- import, export, etc. -- de contenu entre différents systèmes
	  reconnaissant ce format là. Nous l'utiliseront car c'est un des format à
	  partir duquel Moodle sait importer un cours.
	}
}

\newglossaryentry{paas}
  {
	name=PaaS,
	symbol=PaaS,
	description=
	{
	  ({\bf P}latform {\bf a}s {\bf a} {\bf S}ervice). Service cloud offrant une
	  plateforme et plusieurs solutions logicielles accessible à distance. Il
	  suit la logique des service fournit via le cloud comme les SaaS ou les
	  IaaS.
    }
}

\newglossaryentry{ide}
  {
	name=IDE,
	symbol=IDE,
	description=
	{
	  ({\bf I}ntegrated {\bf D}evelopment {\bf E}nvironment). Logiciel facilitant
	  la création et la gestion de code source, le plus souvent au sein de
	  projets correspondant à un logiciel cible particulié.
	}
  }

\newglossaryentry{saas}
  {
	name=SaaS,
	symbol=SaaS,
	description={
	  ({\bf S}oftware {\bf a}s {\bf a} {\bf S}ervice). Logiciel disponible de
	  manière distante via un service de cloud computing. Le entrées et sorties
	  de l'application transitent à travers le réseau entre le client et le
	  serveur hôte du logiciel. Tout le calcul est effectué à distance et est
	  transparent pour le client.
	}
  }

\newglossaryentry{back-end}
  {
	name=Back end,
	symbol=back end,
	description={
	  Pour un logiciel informatique correspond à la partie qui fait office
	  d'interface entre le logiciel et les applications et bibliothèques tierces
	  qui sont invisible pour l'utilisateur -- à l'opposé du \gloss{front-end}.
	}
  }

\newglossaryentry{front-end}
  {
	name=Front end,
	symbol=front end,
	description={
	  Pour un logiciel informatique correspond à la partie qui fait office
	  d'interface entre le corps du logiciel et l'utilisateur. Ça correspond le
	  plus souvent à l'interface utilisateur (graphique ou non) -- à l'opposé du
	  \gloss{back-end}.
	}
  }

\newglossaryentry{ent}
  {
	name=ENT,
	symbol=ENT,
	description={
	  ({\bf E}space {\bf N}umérique de {\bf T}ravail). 
	  Plateforme de travail qui permet aux enseignant de diffuser des contenus
	  de cours ou des exercices aux étudiants en fonction de leur groupe, de
	  leur options, etc.
	}
  }

\newglossaryentry{lms}
  {
	name=LMS,
	symbol=LMS,
	description={
	  ({\bf L}earning {\bf M}anagement {\bf S}ystem). Equivalent d'ENT.
	}
  }

\newglossaryentry{webdav}
  {
	name=WebDAV,
	symbol=WebDAV,
	description={
	  ({\bf Web D}istributed {\bf A}uthoring and {\bf V}ersioning) est une
	  extension du protocole HTTP qui facilite le partage et l'édition
	  collaborative entre utilisateurs. Il permet de faire des opérations de
	  création/édition/suppression de fichiers sur des dépôts partagés par
	  plusieurs utilisateurs tout en gérant les différentes versions de ces
	  documents au moyens de vérrous. Il permet également une gestion des droits
	  d'accès aux fichiers.
	}
  }

\newglossaryentry{zip}
  {
	name=ZIP,
	symbol=zip,
	description={
	  est un format d'archive normalizé contenant une arborescence de
	  fichiers/répertoires compréssés selon un algorithme de compression sans
	  perte de données.
	}
  }

\newglossaryentry{mock}
  {
	name=Mock objects,
	symbol=mock objects,
	description={
	  (litéralement faux objets). Ce sont des objets donc leur comportement
	  (corps de fonction, valeurs retournées, valeurs des attributs, etc.) sont
	  entièrement simulés pour tester des intéractions avec d'autre objets réel
	  en atttendant des résultats connus puisque simulés sans pour autant avoir
	  à connaître une implémentation de ces faux objets.
	}
  }


