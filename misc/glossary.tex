\newglossaryentry{mbd}
  {
	name=MDB,
	symbol=MDB,
	description=
	{
	  {\bf M}odel {\bf B}ased {\bf D}esign. Utiliser notamment dans des application
	  automobiles, aérospatiale ou pour de l'informatique industrielle, MDB est un
	  méthode mathématique et visuelle pour comprendre et résoudre des problème
	   complexes de système de contrôle, de traitement du signal ou des systèmes
	   communicants. Elle est beaucoup utilisée pour des logiciels embarqués.
	}
}
\newglossaryentry{rtw}
  {
	name=RTW-EC\up{\circledR},
	symbol=RTW-EC\up{\circledR},
	description=
	{
	  {\bf R}eal-{\bf T}ime {\bf W}orkshop-{\bf E}mbedded {\bf C}oder\up{\circledR}.
	  Aujourd'hui renommé Simulink Coder\up{\texttrademark}. Outil de génération de code
	  C et C++ à partir de diagrammes Simulink\up{\circledR}. Le code généré peut-être
	  utiliser pour des applications temps réel et hors temps réel, notamment pour le
	  prototypage rapide.
	}
}
\newglossaryentry{rtos}
  {
	name=OS temps réel,
	symbol=RTOS,
	description=
	{
	  ({\bf R}eal {\bf T}ime {\bf O}perating {\bf S}ystem en anglais). Système d'exploitation
	  multi-tâches destinée aux applications temps réel. Il facilite la création d'un système
	  temps réel via des ordonnances de tâches spécialisés afin de fournir aux développeurs
	  des systèmes temps réel les outils et les primitives nécessaires pour produire un
	  comportement temps réel souhaité dans le système final.
	}
}
\newglossaryentry{orianne}
  {
	name=Orianne,
	symbol=Orianne,
	description=
	{
	  //TODO <acronym>. Plateforme de prototypage rapide de fonctions de contrôle moteur. Voir \ref{sec:orianne} pour plus de détails.
	}
}
\newglossaryentry{qgen}
  {
	name=QGen,
	symbol=QGen,
	description=
	{
	  Générateur open source de code embarqué temps réel en C et Ada à partir de modèles
	  Matlab\up{\textregistered} Simulink\up{\textregistered}. Basé sur
	  \gloss{projectp}, son développement est piloté par la société AdaCore.
	}
}
\newglossaryentry{projectp}
  {
	name=Project P,
	symbol=Project P,
	description=
	{
	  Project open source visant à réduire les coût des approches de
	  développement \gloss{mbd} dans le cadre d'applications temps réel
	  embarquées. Le projet comprend de grands groupe industriel comme de plus
	  petites structure qui collaborent dans un but commun. Pour en savoir plus
	  {\tt http://www.open-do.org/projects/p/}
	}
}
