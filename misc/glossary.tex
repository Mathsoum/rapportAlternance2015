\newglossaryentry{mbd}
  {
	name=MBD,
	symbol=MBD,
	description=
	{
	  {\bf M}odel {\bf B}ased {\bf D}esign. Utilisé notamment dans des applications
	  automobiles, aérospatiales ou pour de l'informatique industrielle, MBD est un
	  méthode mathématique et visuelle pour comprendre et résoudre des problèmes
	   complexes de système de contrôle, de traitement du signal ou de système
	   communicant. Elle est beaucoup utilisée pour des logiciels embarqués.
	}
}
\newglossaryentry{rtw}
  {
	name=RTW-EC\up{\circledR},
	symbol=RTW-EC\up{\circledR},
	description=
	{
	  {\bf R}eal-{\bf T}ime {\bf W}orkshop-{\bf E}mbedded {\bf C}oder\up{\circledR}.
	  Aujourd'hui renommé Simulink Coder\up{\texttrademark}. Outil de génération de code
	  C et C++ à partir de diagrammes Simulink\up{\circledR}. Le code généré peut-être
	  utilisé pour des applications temps réel et hors temps réel, notamment pour le
	  prototypage rapide.
	}
}
\newglossaryentry{rtos}
  {
	name=OS temps réel,
	symbol=RTOS,
	description=
	{
	  ({\bf R}eal {\bf T}ime {\bf O}perating {\bf S}ystem en anglais). Système d'exploitation
	  multi-tâches destinée aux applications temps réel. Il facilite la création d'un système
	  temps réel via des ordonnances de tâches spécialisés afin de fournir aux développeurs
	  des systèmes temps réel les outils et les primitives nécessaires pour produire un
	  comportement temps réel souhaité dans le système final.
	}
}
\newglossaryentry{inca}
  {
	name=INCA,
	symbol=INCA,
	description=
	{
	  Logiciel de mesure et de calibration d'\gloss{ecu} via plusieurs protocoles de communication
	  développé par ETAS. Il permet de modifier les calibrations d'une \gloss{ecu} et de vérifier
	  que les changements au niveau des mesures sont corrects.
	}
}
\newglossaryentry{orianne}
  {
	name=Orianne,
	symbol=Orianne,
	description=
	{
	  {\bf O}util numé{\bf RI}que pourme m{\bf A}quettage de fo{\bf N}ctions de co{\bf N}trole mot{\bf E}eur.
	  Plate-forme de prototypage rapide de fonctions de contrôle moteur développée par Aboard Engineering. Voir \ref{sec:orianne}
	  pour plus de détails.
	}
}
\newglossaryentry{qgen}
  {
	name=QGen,
	symbol=QGen,
	description=
	{
	  Générateur open source de code embarqué temps réel en C et Ada à partir de modèles
	  Matlab\up{\textregistered} Simulink\up{\textregistered}. Basé sur
	  \gloss{projectp}, son développement est piloté par la société AdaCore.
	}
}
\newglossaryentry{projectp}
  {
	name=Project P,
	symbol=Project P,
	description=
	{
	  Projet open source visant à réduire les coût des approches de
	  développement \gloss{mbd} dans le cadre d'applications temps réel
	  embarquées. Le projet comprend de grands groupe industriel comme de plus
	  petites structure qui collaborent dans un but commun. Pour en savoir plus
	  {\tt http://www.open-do.org/projects/p/}
	}
}
\newglossaryentry{ecu}
  {
	name=ECU,
	symbol=ECU,
	description=
	{
	  {\bf E}ngine {\bf C}ontrol {\bf U}nit. Le boitier de contrôle moteur contenant des
	  pilotes et API pour manipuler les entrées sorties du boitier. L'application de contrôle
	  moteur sera déployée dans ce boitier et utilisera les pilotes de ce matériel pour interagir avec le moteur.
	}
}
\newglossaryentry{rte}
  {
	name=RTE,
	symbol=RTE,
	description=
	{
	  {\bf R}eal-{\bf T}ime {\bf E}nvironment. Dans une architecture
	  \gloss{autosar}, la couche RTE interface l'application (les fonctions de
	  contrôle moteur) du microgiciel (pilotes et API) embarqué dans le
	  calculateur. Il s'occupe aussi de l'échange de données entre fonctions de
	  l'application car la norme AUTOSAR interdit la communication directe entre
	  fonctions.
	}
}
\newglossaryentry{autosar}
  {
	name=AUTOSAR,
	symbol=AUTOSAR,
	description=
	{
	  {\bf AU}tomotive {\bf S}ystem {\bf AR}rchitecture. C'est une association
	  internationale de développement créée en 2003 dans le but de développer et
	  d'établir une architecture logicielle standardisée et ouverte pour les
	  véhicules. Elle a définie les sépcificaitons AUTOSAR qio sont devenues un
	  standards dans le développement embarqué automobile.
	}
}
\newglossaryentry{can}
  {
	name=CAN,
	symbol=CAN,
	description=
	{
	  Le bus CAN ({\bf C}ontroller {\bf A}rea {\bf N}etwork) est un bus système série
	  très répandu dans beaucoup d'industries, notamment l'automobile. L'approche
	  bus permet de faire du multiplexage plutôt que de la communication point-à-point.
	}
}
\newglossaryentry{canalyzer}
  {
	name=CANalyzer,
	symbol=CANalyzer,
	description=
	{
	  Logiciel permettant de visualiser les trames émises et reçues sur un bus CAN développé par Vector. 
	}
}
\newglossaryentry{osgi}
  {
	name=OSGi,
	symbol=OSGi,
	description=
	{
	  {\bf O}pen {\bf S}ervices {\bf G}ateway {\bf i}nitiative est une plateforme de services
	  fondée sur le langage Java qui permet de créer des composants dynamiques fournissant
	  des services. Ces composants peuvent se connecter et se déconnecter \og à la volée
	  \fg{} à la plateforme.
	}
}
\newglossaryentry{maven}
  {
	name=Maven,
	symbol=Maven,
	description=
	{
	  Maven est un outil pour la gestion de l'automatisation de production des projets logiciels
	  Java. L'objectif recherché est de produire un logiciel à partir de se sources en optimisant
	  les tâches réalisées à cette fin et en garantissant le bon ordre de fabrication.
	}
}
\newglossaryentry{a2l}
  {
	name=A2L,
	symbol=A2L,
	description=
	{
	  est un format de fichier relatif au standard ASAM MCD-2 MC (ou ASAP2). Ce
	  standard définit un format de description pour les variables internes
	  d'une ECU utilisée pour les mesures et la calibration. Les fichiers A2L
	  facilite l'accès des utilisateurs aux paramètres internes d'une ECU grâce
	  à des noms symboliques. Ces fichiers sont ensuite utilisés par des
	  logiciels tiers pour visualiser ou modifier ces mesures et calibrations.
	}
}
