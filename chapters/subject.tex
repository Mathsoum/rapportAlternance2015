\chapter{Présentation du sujet}
Ayant intégré une équipe travaillant sur un projet existant, je n'ai pas réalisé
un travail de mon côté mais plutôt ajouté de nouvelles fonctionnalités à ce
projet. C'est pourquoi ce chapitre reprend les différentes fonctionnalités que
j'ai développé après une petite introduction du projet global.
\section{Elaastic}
Elaastic est un outil pour concevoir et diffuser des contenus pédagogiques. Il
permet de concevoir des supports pédagogiques structurés, interactifs et
indépendants du format de destination. Il offre la possibilité de diffuser ces
supports sous forme de diaporama ou sous forme de publication au format livre ou
au format site intéractif dans un ENT ou un LMS.\\

D'un point de vue technique, Elaastic se découpe en une partie Grails pour la
corps et la partie \gloss{back-end}. Le système de persistence des données est
assurée par Mongo pour la version de développement. La version déployée en
production s'appuie sur une base de données Postgre SQL. Côté \gloss{front-end},
l'interface utilisateur est conçu grace à AngularJS, un framework Javascript.

\begin{figure}[h]
  \centering
  \includegraphics[scale=0.4]{images/elaastic_blue.pdf}%
  \caption{Logo d'Elaastic}
  \label{fig:elaastic}
\end{figure}

Pour ma part, j'ai essentiellement travaillé sur le corps et la partie back end.
Les sections qui suivent détaillent les fonctionnalités sur lesquelles j'ai
travaillé.

\subsection{IMS}
\subsection{WebDAV}
\subsection{Artifact}
\subsection{Diffusion}

\section{Démarche méthodologique} %TODO Rephrase
