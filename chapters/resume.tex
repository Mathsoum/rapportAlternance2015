\chapter*{Résumé}
\addcontentsline{toc}{chapter}{Résumé}

\paragraph*{Mots-clés}
Matlab, Langage C, Java Swing, MBD, Génération de code, Embarqué, Communicaiton
CAN, Automobile, Optimisation, Factorisation.

\paragraph*{}
J'ai effectué ce stage dans le cadre du Master 2 Développement Logiciel. Il
s'est étalé sur 9 mois du 3 novembre 2014 au 31 juillet 2015 en alternance au
rythme de 3 jours par semaine en entreprise et 2 jours à l'université. J'ai
rejoint Aboard Engineering, une entreprise experte dans les domaines de
l'électronique et de l'informatique industrielle. J'ai occupé un poste
d'ingénieur en développement logiciel et j'ai travaillé sur deux sujets en
parallèle : la plateforme Orianne et le générateur de code QGen.

La plateforme permet de faire du prototypage rapide de fonctions de contrôle
moteur à partir de code source généré par Matlab\up{\circledR}
RTW-EC\up{\circledR}. Elle est en cours de développement. J'ai donc repris du
code existant que j'ai factorisé pour des raisons de clarté et de complexité.
J'ai ensuite ajouté des outils manquants à cette plateforme comme un outil de
génération de code pour de la communication CAN. Une ébauche de l'outil existait
mais le code généré était trop gourmand en terme de mémoire pour être intégré
dans une application de contrôle moteur.

Le générateur QGen est lui aussi en cours de développement mais pas par Aboard
Engineering. QGen est un générateur open-source de code embarqué temps réel en C
et Ada. Le projet est piloté par AdaCore et est issu d'un autre projet
open-source baptisé Project P. Le rôle que joue Aboard Engineering dans
l'évolution de ce générateur relève du test et de la revue. Nous possédons les
qualifications techniques pour analyser le code généré par QGen. Notre objectif
était de comparer en terme de temps d'exécution et d'impact mémoire les
performances du code généré par QGen par rapport au code généré par
Matlab\up{\circledR} RTW-EC\up{\circledR}, outil que nous utilisons et que nous
avons pris pour référence de comparaison sur ces points précis.

Au terme de ce stage, les objectifs sont remplis. Le générateur CAN est
fonctionnel et les résultats de l'amélioration du code généré sont concluants.
En effet, j'ai pu réduire de manière significative l'impact en mémoire (7 fois
moins d'espace utilisé) tout en gardant un outil opérationnel. J'ai aussi
développé une partie de configuration graphique pour s'intégrer au reste de la
plateforme. Je travaille actuellement à la conception et au développement d'un
outils de génération de code faisant le lien entre l'application en question et
le matériel sur lequel elle sera déployée.

Les résultats des tests obtenus via la collaboration avec AdaCore autour de QGen
sont eux aussi très concluants. Nous sommes arrivé à obtenir des performances
équivalentes entre le code généré par RTW-EC\up{\circledR} et QGen. Une première
version majeure doit sortir courant juin 2015. Avec cette version finale, nous
devrions être en mesure d'effectuer des tests plus poussés sur banc de test et,
si ces derniers sont validés, des tests directement sur véhicule.



