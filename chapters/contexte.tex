\chapter{Contexte}
%TODO Something here ?
\section{Ticetime}
\newcommand{\ticetime}[2]{\item[{\it #1}] #2}
Ticetime est une jeune entreprise créée en 2011 et composée aujourd'hui de deux salariés :
\begin{itemize}
  \ticetime{Franck Silvestre}{Directeur général. Il occupe également
  un poste de maître de conférence associé à l'Université Paul Sabatier de
  Toulouse}
  \ticetime{John Tranier}{Directeur technique. Il donne également des cours à
  l'Université Montpellier II.}
\end{itemize}

\paragraph{Historique}
\begin{description}
  \item[Juin 2009] Franck occupe la fonction de directeur technique au sein
	d'OMT-Fylab et recrute John Tranier comme architecte sur le projet Open ENT
	- Lilie. Cette rencontre marque le début d'une expérience professionnelle
	unique et d'une amitié.
  \item[Janvier 2011] Création de la société Ticetime par Franck Silvestre après
	son départ d'OMT/Fylab. Franck travaille comme consultant et développeur
	indépendant principalement sur le projet TD Base, brique pédagogique de
	l'Open ENT.
  \item[Janvier 2012] Franck intègre à mi-temps l'Université Toulouse III en
	qualité de maître de conférences associé. L'idée du projet elaastic et son
	premier prototype germe devant la frustration rencontrée lors de la
	conception des supports de cours.
  \item[Juin 2013] John Tranier intègre comme associé et directeur technique
	Ticetime pour participer au développement de la société et pour prendre en
	charge la mise en oeuvre industrielle d'elaastic. Ticetime, partenaire de
	Setec IS, accompagne la région PACA dans la mise en oeuvre de son
	Environnement Numérique Éducatif.
\end{description}
\section{L'équipe}
Durant ce stage, j'ai eu l'occasion de travailler avec :\\
\begin{itemize}
  \ticetime{Franck Silvestre}{directeur général.}
  \ticetime{John Tranier}{directeur technique.}
  \ticetime{Vincent Tertre}{alternant en Master 2 DL à Paul Sabatier.}
\end{itemize}
\section{Les outils utilisés}
J'ai développé dans un environnement Fedora 20 (Heinsenbug).
Nous travaillions avec divers outils dont certains distants (précisés dans la
liste ci-après ; cette liste comprend aussi les frameworks logiciels utilisés).
\begin{description}
  \item[CloudBees] \gloss{paas} qui embarque le système d'intégration continu
	Jenkins et qui permet le déploiement automatique d'application web et
	mobile.
  \item[CodeNarc] Analyseur de code spécialisé pour Groovy.
  \item[Git] Gestionnaire de version.
  \item[Grails] Framework d'application web utilisant le langage Groovy. Il suit
	le paradigme \og codage par convention \fg{} masquant certains
	configurations au développeur.
  \item[GVM] (the {\bf G}roovy en{\bf V}ironnement {\bf M}anager) Outil qui
	permet l'installation d'autres outils relatif à un environnement
	Groovy/Grails à des versions particulières et de les mettres à jour
	simplement.
  \item[IntelliJ IDEA] \gloss{ide} avec lequel j'ai rédigé tout le code source
	que j'ai produit. La gestion des projets et des environnements Grails
	apporte beaucoup de confort de travail. Nous utilisions l'outil de
	vérification de couverture de code intégré à l'IDE.
  \item[Jira] \gloss{saas} de gestion de projet et de gestion des bugs/incidents. Il
	fournit une interface \og tableau Kanban \fg{} que nous avons beaucoup utilisé.
  \item[Spock] Un framework de test et de spécification pour les applications
	Java et Groovy.
  \item[UMongo] Application permettant d'explorer et de modifier une base de
	donnée Mongo. Mongo est une base de données NoSQL regroupant des collections
	de documents au format Json.
\end{description}
