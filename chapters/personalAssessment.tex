\chapter{Bilan}

\section{Expérience professionnelle}
Si j'ai choisi ce stage, c'est pour avoir une expérience professionnelle dans le
web. Durant mon parcours je n'ai pas eu l'occasion de participer à un projet web
d'envergure autre que ceux fait à l'université. Il m'a permis de cerner les
enjeux et les contraintes lors de la réalisation d'un projet web. J'ai aussi eu
un aperçu des éventuelles évolutions par lesquelles il peut passer et qu'il faut
prendre en compte lors de la conception d'une nouvelle fonctionnalité.

Cette expérience m'a permis de voir l'organisation d'un framework web comme
Grails et des avantages et inconvénients que cela implique d'utiliser une telle
technologie pour son application web.

J'ai également pu pratiquer le développement en suivant une approche agile pour
un projet professionnel concret et au sein d'une équipe de développement qui
maîtrise ces pratiques. J'ai là encore pu voir les avantages que cela apporte au
projet mais aussi parfois une certaine contrainte d'organisation et de rigueur
pour les développeurs, nécessaire au bon fonctionnement de ces méthodes.

\section{Apports personnels}
D'un point de vue plus personnel, je n'avais fait du web que pour des projets
personnels qui sa\-tisfaisaient plus ma curiosité que vraiment une nécessité
technique ou fonctionnelle. Ils n'allaient donc par conséquent pas très profond
dans la technique et ne suivaient pas forcément les bonnes pratiques de
développement. Une autre de mes motivations est que nous ne faisons pas
énormément de web à l'université alors qu'on en entend parler de plus en plus au
niveau professionnel. Des entreprises se spécialisent dans le web, fournissent
des services via internet. L'essor du \og as a service \fg{} ne fait aucun doute
et le développement utilisant une puissance de calculs et des outils
dématérialisés se popularise. Il était donc important pour moi de me renseigner
sur ce sujet et c'est pourquoi j'ai répondu à cette offre de stage.\\

Au terme de ces trois mois de stage, j'ai pu faire un point sur cette expérience
dans mon parcours. Certes il tend à compléter mon CV avec une expérience dans le
web mais il m'a aussi conforté dans mon idée que j'ai une préférence pour les
applications de type client lourd local -- une application dont le calcul est
effectué en local -- plutôt que client léger dématérialisé -- une application
dont la partie calcul est dématérialisée et utilisée depuis un client (léger)
distant.  Je conçois la puissance que sont par exemple les services REST et je
ne les rejette pas complètement. Cependant, je garde une grande affection pour
les langages natifs comme le C++ que j'ai eu l'occasion de pratiquer beaucoup
durant mon parcours et que j'ai tendance à privilégier pour mes projets
personnels.


