\chapter{Bilan}

\section*{Expérience professionnelle}
Au cours de mes expériences professionnelles précédentes, j'ai eu l'occasion de voir plusieurs domaines,
plusieurs technologies, plusieurs structures d'entreprise et plusieurs méthodes de travail. J'ai donc eu l'occasion
de faire la part des choses et identifier les aspects que j'aime et ceux que je n'aime pas.

Les stages sont un très bon moyen de découvrir des environnement professionnels différents. C'est pourquoi
j'ai été tenté par un stage dans l'automobile, un domaine que je n'avais pas encore exploré. De plus, je me sens
bien dans les petites structures où on a généralement plus le
droit à la parole et où notre avis compte en terme de technique et de solution à adopté. Je retiens des discussions que
j'ai eu avec mes camarades qu'il n'est pas toujours évident de défendre ses idées en terme de conception ou
de technologie quand on est obligé de suivre des politiques commune à plusieurs équipes comme c'est le cas dans
certaines grandes structures.

Ce stage constitue une expérience pour moi dans le domaine de l'informatique industrielle embarquée et plus
particulièrement dans l'automobile. Il m'a aussi permis de découvrir le MBD dont on nous a parlé en cours et
que je souhaitais voir dans un contexte professionnel pour voir les contraintes que cela pose et leur impact
sur un projet concret.

Sans oublier le format du stage qui est un point essentiel de cette expérience. Le fait d'effectuer ce stage en alternance
m'a permis de suivre un projet de longue haleine et d'y participer pleinement.

D'un point point de vue plus objectif, je dirais que j'ai rempli mes objectifs de stage.
J'ai réussi les missions qui m'ont été confiées et je suis satisfait du travail que j'ai effectué durant ces 9 mois.
J'ai travaillé avec une combinaison de plusieurs technologies, utiliser des compétences différentes et même
appris de nouvelles. Et ce, pour atteindre la même finalité. Je trouve cela très enrichissant et je pense que ça
nous permet de mieux appréhender les problème en abordant des solutions sous plusieurs angles.

\section*{Apports personnels}
D'un point de vue plus personnel, je me suis conforté dans l'idée que je suis intéressé par l'informatique embarqué.
Ce stage a aussi éveillé ma curiosité dans le développement plus orienté système.
Je ne pense pas que je souhaiterais faire toute ma carrière dans ce domaine technique mais c'est ce qui m'intéresse actuellement.

J'ai pu aussi me conforté dans mes compétences en anglais. Toute la communication avec AdaCore autour de QGen s'est faite
en anglais via leur gestionnaire de bug ou par e-mail. Je me sens à l'aise dans la communication en anglais et je songe
à chercher du travail à l'étranger pour enrichir encore mon vocabulaire et découvrir d'autres rythmes de vie.

J'ai récemment commencé ma recherche d'emploi pour après ce stage. J'ai plusieurs opportunités dans des domaines différents.
Je suis actuellement en discussion avec Aboard Engineering à propos d'un poste de développeur logiciel et système.
Je recherche des offres qui me permettraient d'approfondir mes compétences technique et pourquoi pas me familiariser
avec de nouvelles en les mettant en application dans des projets concrets.


