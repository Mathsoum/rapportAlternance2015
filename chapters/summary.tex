\chapter*{Summary}
\addcontentsline{toc}{chapter}{Summary}

\section*{Key words}
\begin{itemize}
  \item Matlab
  \item C language
  \item Java Swing
  \item MBD
  \item Source code generation
  \item Embedded computing
  \item CAN communication
  \item Automotive
  \item Optimisation
  \item Factoring
\end{itemize}

\paragraph*{}
This internship took place during the last year of my Master's Degree in Software Development. It was 9 months long, from November, 3\up{rd} 2014 to July, 31\up{th} 2015. It was a work-study internship with 3 days a week at work and 2 days a week at university. I joined Aboard Engineering, a service company expert in electronics and industrial computing. I was a software designer engineer and I worked on 2 subjects at the same time: the Orianne platform and the QGen source code generator.

The platform allows to do rapid prototyping for engine management functions development based on source code generated using Matlab\up{\circledR} RTW-EC\up{\circledR}.
It is currently in development. So I used already written code that I factored for clarity and complexity reasons. Then I added some missing tools to the platform like a code generator for CAN communication. A first draft of the tool existed but it was taking too much memory space to be integrated in an engine control application.

Le QGen generator was in development too but not by Aboard Engineering. It is an open source real time source code generator. The project is driven by AdaCore and comes from an other open source project called Project P. The role played by Aboard Engineering in the evolution of that tool is about tests and reviews. We have the technical qualifications to analyze the source code generated by QGen. Our objective was to compare the code generated by QGen and the one generated by Matlab\up{\circledR} RTW-EC\up{\circledR} regarding execution time and memory usage. Matlab\up{\circledR} RTW-EC\up{\circledR} is a tool that we used at Aboard Engineering so we took it as a benchmark for our tests and comparisons on those particular points.

At the end of the internship, the objectives are completed. The CAN generator is functional and the results from the improvements of the generated code are conclusive. In fact, I achieve to reduce in a significant way the memory usage (7 times less memory space used) without loosing features. I also developed a graphical interface for the tool configuration to be integrated in the rest of the platform. I currently work on the design and development of another tool that will generate the code to bind the application en the hardware in which it will be deployed.

The results from the tests about QGen obtained in collaboration with AdaCore are also conclusive. We achieved to obtain similar performance between the code generated by RTW-EC\up{\circledR} and the one generated bu QGen. A first major version would be released during June 2015. With this final version, we should be able to do further tests on test bench and, if those are validated, more tests directly on a vehicle.





