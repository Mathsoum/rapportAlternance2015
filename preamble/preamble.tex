
\usepackage[frenchb]{babel}
\usepackage[utf8]{inputenc}
\usepackage[T1]{fontenc}
\usepackage{lipsum}
\usepackage{graphicx}
\usepackage[left=3.5cm, right=2cm, top=2cm, bottom=2.5cm]{geometry}
\usepackage{listings}
\usepackage[hidelinks]{hyperref}
\usepackage{amsmath,amsfonts,amssymb}
\usepackage{tikz}
\usepackage{shorttoc}
\usepackage{makeidx}
\usepackage[nopostdot,nonumberlist,toc]{glossaries}
\usepackage{nomencl}
\usepackage{tabularx}
 \usepackage{ifthen}
\usepackage[Lenny]{fncychap}

\author{Mathieu SOUM}

\makeatletter
\let\thetitle\@title
\let\theauthor\@author
\makeatother

\newcommand{\bigtitle}{Développement, amélioration et intégration d'outils de génération de code embarqué}
\newcommand{\shorttitle}{pour une plateforme de prototypage rapide fonctions de contrôle moteur}
\newcommand{\tuteur}{Sébastien RICHE}
\newcommand{\encadrant}{Isabelle FERRAN\'E}
\newcommand{\coverfoot}{Master 2 -- Développement Logiciel / 2014 - 2015}

\newcommand{\coverpage}[1]{%
\newgeometry{top=2cm,left=2cm,bottom=2cm,right=1cm}
\begin{titlepage}
\parindent=0pt
\begin{tabular}[h]{l@{ : }l}
  Tuteur & \tuteur \\
  Encadrant & \encadrant
\end{tabular}
\ifthenelse{\equal{#1}{cover}}%
{\vspace{1.5cm}}%
{\vspace{3.5cm}}
\begin{center}
\ifthenelse{\equal{#1}{cover}}
{
  \includegraphics[scale=0.9]{images/aboard.png}%
}{}
\end{center}
\vspace{2cm}
\hrulefill
\begin{center}\bfseries\Huge
  \bigtitle
\end{center}
\hrulefill
\begin{center}\bfseries\Large
  \shorttitle
\end{center}
\vspace*{0.5cm}
\begin{center}\bfseries\Large
  \theauthor
\end{center}
\vspace*{0.2cm}
\begin{center}\bfseries
  Stage du 3 novembre 2014 au 31 juillet 2015\\Chez Aboard Engineering
\end{center}
\vfill
\ifthenelse{\equal{#1}{cover}}
{
\begin{tabular}[h]{m{4.5cm}m{5.7cm}m{5.7cm}}
\includegraphics[scale=0.45]{images/aboard.png} &
\includegraphics[scale=0.55]{images/ups.png} &
\includegraphics[scale=0.20]{images/mdl.png} \\
\end{tabular}
\vspace{1.2cm}
}{}
\begin{flushright}
  \coverfoot
\end{flushright}
\end{titlepage}
\restoregeometry
}

\makeindex

%\fancyhead[RE]{\rightmark}
%\fancyhead[LO]{\leftmark}
%\fancyhead[RO,LE]{}
%\fancyfoot[LE,RO]{\thepage}
%\fancyfoot[C]{}

\usepackage{fancyhdr}%pour les en-têtes et pieds de pages
        \setlength{\headheight}{14.2pt}% hauteur de l'en-tête
%%%%%%%%%%%%%%%%%%%style front%%%%%%%%%%%%%%%%%%%%%%%%%%%%%%%%%%%%%%%%% 
        \fancypagestyle{front}{%
                \fancyhf{}%on vide l'en-tête
                \fancyfoot[C]{\thepage}%
                \renewcommand{\headrulewidth}{0pt}%trait horizontal pour l'en-tête
                \renewcommand{\footrulewidth}{0pt}%trait horizontal pour le pied de page
                }
%\renewcommand{\chaptermark}[1]{%
%\markboth{\MakeUppercase{%
%\chaptername}\ \thechapter.%
%\ #1}{}}
%%%%%%%%%%%%%%%%%%%style main%%%%%%%%%%%%%%%%%%%%%%%%%%%%%%%%%%%%
        \fancypagestyle{main}{%
                \fancyhf{}
                \renewcommand{\headrulewidth}{0pt}%trait horizontal pour l'en-tête
                \renewcommand{\footrulewidth}{0pt}%trait horizontal pour le pied de pages
                \renewcommand{\chaptermark}[1]{\markboth{\chaptername\ \thechapter.\ ##1}{}}% redéfinition pour avoir ici les titres des chapitres des sections en minuscules
                \renewcommand{\sectionmark}[1]{\markright{\thesection\ ##1}}
                \fancyhead[C, RE, LO]{}
                \fancyhead[LE]{\rightmark}%
                \fancyhead[RO]{\leftmark}
                \fancyfoot[C]{}
                \fancyfoot[RO,LE]{\thepage}%
                \fancyfoot[LO,RE]{}
                }
%%%%%%%%%%%%%%%%%%%style back%%%%%%%%%%%%%%%%%%%%%%%%%%%%%%%%%%%%%%%%%  
        \fancypagestyle{back}{%
                \fancyhf{}%on vide l'en-tête
                \fancyfoot[C]{\thepage}%
                \renewcommand{\headrulewidth}{0pt}%trait horizontal pour l'en-tête
                \renewcommand{\footrulewidth}{0pt}%trait horizontal pour le pied de pages
                }

\newcommand{\gloss}[1]{{\sl \glssymbol{#1}}}
\newcommand{\Gloss}[1]{{\sl \Glssymbol{#1}}}

\newglossaryentry{imscp}
  {
	name=Paquetage de Contenu IMS,
	symbol=Paquetage de Contenu IMS,
	description={est un format de paquet de contenu créé par l'IMS Global Learning Consortium. C'est
un format standardisé qui permet l'échange -- import, export, etc. -- de contenu
entre différents systèmes reconnaissant ce format là. Nous l'utiliseront car
c'est un des format à partir duquel Moodle sait importer un cours.}}

\newglossaryentry{paas}
  {
	name=PaaS,
	symbol=PaaS,
	description={
	  ({\bf P}latform {\bf a}s {\bf a} {\bf S}ervice) Service cloud
  offrant une plateforme et plusieurs solutions logicielles accessible à
distance. Il suit la logique des service fournit via le cloud comme les SaaS ou
les IaaS.}
}

\newglossaryentry{ide}
  {
	name=IDE,
	symbol=IDE,
	description={
	  ({\bf I}ntegrated {\bf D}evelopment {\bf E}nvironment) Logiciel facilitant
	  la création et la gestion de code source, le plus souvent au sein de
	  projets correspondant à un logiciel cible particulié.
	}
  }

\newglossaryentry{saas}
  {
	name=SaaS,
	symbol=SaaS,
	description={
	  ({\bf S}oftware {\bf a}s {\bf a} {\bf S}ervice) Logiciel disponible de
	  manière distante via un service de cloud computing. Le entrées et sorties
	  de l'application transitent à travers le réseau entre le client et le
	  serveur hôte du logiciel. Tout le calcul est effectué à distance et est
	  transparent pour le client.
	}
  }

\newglossaryentry{back-end}
  {
	name=Back end,
	symbol=back end,
	description={
	  pour un logiciel informatique correspond à la partie qui fait office
	  d'interface entre le logiciel et les applications et bibliothèques tierces
	  qui sont invisible pour l'utilisateur -- à l'opposé du \gloss{front-end}.
	}
  }

\newglossaryentry{front-end}
  {
	name=Front end,
	symbol=front end,
	description={
	  pour un logiciel informatique correspond à la partie qui fait office
	  d'interface entre le corps du logiciel et l'utilisateur. Ça correspond le
	  plus souvent à l'interface utilisateur (graphique ou non) -- à l'opposé du
	  \gloss{back-end}.
	}
  }

\newglossaryentry{ent}
  {
	name=ENT,
	symbol=ENT,
	description={
	  {\bf E}sapce {\bf N}umérique de {\bf T}ravail
	}
  }

\newglossaryentry{lms}
  {
	name=LMS,
	symbol=LMS,
	description={
	  {\bf L}earning {\bf M}anagement {\bf S}ystem
	}
  }

\newglossaryentry{webdav}
  {
	name=WebDAV,
	symbol=WebDAV,
	description={
	  ({\bf Web D}istributed {\bf A}uthoring and {\bf V}ersioning) est une
	  extension du protocole HTTP qui facilite le partage et l'édition
	  collaborative entre utilisateurs. Il permet de faire des opérations de
	  création/édition/suppression de fichiers sur des dépôts partagés par
	  plusieurs utilisateurs tout en gérant les différentes versions de ces
	  documents au moyens de vérrous. Il permet également une gestion des droits
	  d'accès aux fichiers.
	}
  }




\makeglossaries
