\chapter*{Introduction}
\addcontentsline{toc}{chapter}{Introduction}
\markboth{Introduction}{}

\section*{Contexte du stage}
\addcontentsline{toc}{section}{Contexte du stage}
Ce stage s'inscrit dans le cadre de ma formation de Master 1 Informatique parcours
Développement Logiciel. Il s'étend sur 3 mois du 5 mai au 1\ier{} août 2014.\\

Durant cette période, j'ai rejoint l'équipe de TiceTime pour travailler sur un
outil d'édition de contenu pédagogique. Cet outil est à destination des
enseignants et a pour but de leur permettre de créer des cours dans un format de
type diaporama et ensuite de l'exporter ou le diffuser dans plusieurs autres formats
(PDF, RTF, Diaporama HTML, \gloss{imscp}).

L'intérêt pour l'entreprise de proposer ce stage était principalement d'avancer
dans le développement de cet outil. Il n'y avait donc pas d'objectif à
proprement dit qu'il faudrait avoir atteint au bout des 3 mois de stage. Les
fonctionnalités à implémenter se sont plutôt ajoutées au fil de l'avancement du
stage et de la réalisation des fonctionnalités précédentes.

\section*{Structure du rapport}
\addcontentsline{toc}{section}{Structure du rapport}

Ce rapport de stage sera découpé en 5 chapitres.\\

Premièrement, un rappel du contexte dans lequel s'est déroulé ce stage. Une
description de l'entreprise, de l'équipe avec laquelle j'ai travaillé et des
différents outils utilisés.

Ensuite, une présentation détaillée du sujet en précisant le détail des
fonctionnalités que j'ai implémentées et de la démarche méthodologique
employée.

Puis, le travail effectivement réalisé. On s'attardera essentiellement sur
l'aspect technique, la description fonctionnellement étant décrite dans le
chapitre précédent.

Enfin, un bilan personnel pour prendre du recul sur cette expérience et la
contextualiser par rapport à ma formation, mon parcours professionnel et mon
futur projet professionnel.

\section*{Conseils de lecture}
\addcontentsline{toc}{section}{Conseils de lecture}

Ce document est à destination de mon maître de stage, mon encadrant
universitaire ainsi que le jury de ma soutenance qui l'aura à disposition. Pour
faciliter sa lecture, voici quelques précisions quant aux détails typographiques
utilisés dans ce document.

Les citations relatives à du code source telles que les noms de classes par
exemple seront écrit avec une {\tt police à chasse fixe}.

Un glossaire à la fin du document répertorie certains termes techniques qui
nécessiteraient une définition plus précise. Les mots apparaissant dans ce
glossaire seront {\sl en écriture penchée} dans le reste du document.
