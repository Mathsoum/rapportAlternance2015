\chapter{Introduction}
%\addcontentsline{toc}{chapter}{Introduction}
\markboth{Introduction}{}

\section*{Contexte du stage}
%\addcontentsline{toc}{section}{Contexte du stage}
\markboth{Introduction}{Contexte du stage}
Ce stage s'inscrit dans le cadre du Master 2 Développement Logiciel à l'Unversité Paul Sabatier de Toulouse. Cette formation a la particularité d'offrir un stage en alternance dès le début de l'année universitaire. C'est une très bonne expérience qui nous tient en haleine tout au long de notre dernière année d'étude et nous permet -- pour la première fois dans notre cursus -- de réaliser des projets d'envergure en entreprise et de pouvoir nous y impliquer pleinement sur une durée plus longue que lors des expériences de stage précédentes.

Ce stage s'est donc déroulé en alternance au rythme de 3 jours par semaine en entreprise (Lundi, Mardi, Mercredi) et 2 jours à l'université (Jeudi, Vendredi) sur une durée totale de 9 mois (3 Novembre 2014 -- 31 Juillet 2015).

\section*{Structure du rapport}
%\addcontentsline{toc}{section}{Structure du rapport}
\markboth{Introduction}{Structure du rapport}
Ce rapport va rendre compte de mon expérience de stage. Il sera découpé selon 3 parties principales :
\begin{description}
	\item[Le contexte] dans lequel j'ai évolué durant les 9 mois de stage. Dans ce chapitre, je ferai une présentation de l'entreprise qui m'a accueillie ainsi que l'environnement de travail dans lequel j'ai évolué.
	\item[Le sujet] sur lequel j'ai travaillé. Je ferai une description en détail de la problématique relative à mon stage et du travail à réaliser.
	\item[Le travail réalisé] durant ce stage. Ce chapitre abordera les détails techniques de mon stage. J'y développerai les différentes tâches que j'ai réalisées et en quoi elles aboutissent à une solution viable à la problématique.
\end{description}

\section*{Conseils de lecture}
%\addcontentsline{toc}{section}{Conseils de lecture}
\markboth{Introduction}{Conseils de lecture}
Ce document est à destination de mon maître de stage, de mon encadrant universitaire ainsi que du jury de ma soutenance qui l'aura à disposition. Pour faciliter sa lecture, voici quelques précisions quant aux détails typographiques utilisés dans ce document.

Les citations relatives à du code source telles que des types de donnée ou des éléments du système de fichier seront écrit avec une {\tt police à chasse fixe}.

Un glossaire à la fin du document répertorie certains termes techniques et acronymes qui nécessiteraient une définition plus précise. Les mots apparaissant dans ce glossaire seront écris {\it italique} lors de leur première apparition dans ce document.
