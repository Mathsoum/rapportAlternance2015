%        File: main.tex
%     Created: jeu. août 21 03:00  2014 C
% Last Change: jeu. août 21 03:00  2014 C
%
\documentclass[a4paper, 11pt]{article}

\usepackage[utf8]{inputenc}
\usepackage[T1]{fontenc}
\usepackage[french]{babel}
\usepackage{lipsum}
\usepackage[top=1cm, bottom=2.5cm, right=2.5cm, left=2.5cm]{geometry}

\title{
  Participation au développement d'une application Java EE dédiée à la
  conception et à la diffusion de support pédagogiques\\
  \vspace{10px}
  {
	\large Stage chez TiceTime\\
	Du 5 mai 2014 au 1 août 2014\\
	Master 1 Informatique -- Développement Logiciel
  }
}

\author{
  Mathieu {\sc Soum}
  \vspace{10px}\\
  Maître de stage\\
	John {\sc Tranier} {\tt john.tranier@ticetime.com}
}

\date{}
\begin{document}

\maketitle

Durant ce stage, j'ai intégré un projet en cours, lancé depuis plusieurs mois et
qui durera plusieurs autres mois après mon départ. L'outil sur lequel j'ai
travaillé s'appelle Elaastic. C'est un outil de rédaction et de diffusion de
contenu pédagogique. Son objectif principal est de permettre aux enseignant de
rédiger leurs cours une seule fois et de pouvoir les exporter dans plusieurs
formats pour les diffuser à ses étudiants directement via Elaastic ou en
utilisant l'ENT de son établissement.\\

TiceTime souhaitait proposer cette offre de stage pour avancer sur ce projet
grâce à de la main d'\oe uvre supplémentaire.
J'ai donc intégré l'équipe qui travaille sur ce projet composée de :
\begin{itemize}
  \item John {\sc Tranier} : Directeur technique chez TiceTime ;
  \item Vincent {\sc Tertre} : Alternant en Master 2 Développement Logiciel à Paul
	Sabatier.
\end{itemize}
\vspace{10px}

Pour ce qui est de la méthodologie et des conditions de travail, j'ai effectué
ce stage à distance depuis chez moi. La raison principale de ce mode de travail
est que TiceTime n'a pas de locaux d'entreprise. L'entreprise comporte
actuellement deux personnes à temps plein. Ces personnes n'étant pas tous les
jours sur Toulouse, les locaux ne seraient pas occupés toute l'année. Nous nous
retrouvions tout de même à La Cantine\footnote{La Cantine est un espace de
coworking au centre de Toulouse.} quand toute l'équipe était sur Toulouse.

D'un point de vue technique, Elaastic est une application web développée en
Groovy avec le framework Grails pour le corps et la partie back end. La partie
front end est développée en Javascript avec le framework AngularJS.

Mon travail sur ce projet a été de développer de nouvelles fonctionnalités pour
améliorer Elaastic et l'amener un peu plus près de son objectif final. J'ai
participé à l'amélioration du système d'exportation de cours en ajoutant des
type d'exports comme l'IMS Content Packaging compatible avec la plateforme
Moodle.

J'ai aussi permis la saisie de nouveaux éléments de cours comme des
questions intéractives pour que les étudiants puissent répondre et comparés
leurs résultats.

Puis, j'ai travaillé sur le système de diffusion pour
permettre aux étudiants d'accéder aux cours rédigés par leur professeur dans les
différents formats d'exportations.

Enfin, j'ai mis en place un système de cache
sur les transformations de cours pour faciclité la récupération de masse par un
ou plusieurs groupe d'étudiant du même document de cours.






\end{document}


