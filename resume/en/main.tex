%        File: main.tex
%     Created: jeu. août 21 03:00  2014 C
% Last Change: jeu. août 21 03:00  2014 C
%
\documentclass[a4paper, 11pt]{article}

\usepackage[utf8]{inputenc}
\usepackage[T1]{fontenc}
\usepackage[english]{babel}
\usepackage{lipsum}
\usepackage[top=1cm, bottom=2.5cm, right=2.5cm, left=2.5cm]{geometry}

\newcommand{\ts}{\textsuperscript}

\title{
  Contribution to the development of a Java EE application dedicated to
  conception an diffusion of teaching material
  \vspace{10px}
  {
	\large Internship at TiceTime\\
	From May, 5\ts{th} 2014 to August, 1\ts{st} 2014\\
	Master's Degree in Information Technology -- Software Development
  }
}

\author{
  Mathieu {\sc Soum}
  \vspace{10px}\\
  Internship supervisor\\
	John {\sc Tranier} {\tt john.tranier@ticetime.com}
}

\date{}
\begin{document}

\maketitle

During that internship, I joined a runing project, launched few months ago and
which will continue a few more once I'll be left. The application on wich I've
worked is called Elaastic. It's a tool for teaching content writting and
diffusion. Its main goal is to allow teachers to write their course once and
export them in several formats to broadcast them to their students directly from
Elaastic or using their university's LMS.\\

TiceTime wanted to propose this offer of internship to progress on this project
using more labor force. So, I joined the team that works on this project
composed with :
\begin{itemize}
  \item John {\sc Tranier} : Technical director at TiceTime ;
  \item Vincent {\sc Tertre} : Student in block-release training for 2\ts{nd}
	year Master's Degree in Software development at Paul Sabatier University.
\end{itemize}
\vspace{10px}

Concerning work methodology and conditions, I made this internship remotly from
home. The main reason for that work model is that TiceTime has no working
office. The company has currently two members full-time. Because thay aren't
often in Toulouse, there's no need to have an office. Besides, we sometimes met
together at La Cantine\footnote{La Cantine is a co-working sapce in
Toulouse town center.} when all the team was in Toulouse.

From a technical point of view, Elaastic is a web application developed in
Groovy using Grails framework for the body and the back end. The Front end is developed
in Javascript using AngularJS framework.

My actual work on this project was to develop new functionalities to improve
Elaastic and bring it toward its final goal. I participate to the improvement of
the course exportation system adding new target formats like IMS Content
Packaging that is compatible with the Moodle platform.

I also allowed the user
to add new elements to a course like interactive questions so the students can
answer them and share their results with their class.

Then, I worked on a
broadcast system to allow students to access their teacher's course in the
severals exportation formats available.

Finally, I set up a cache system on the
course transformations to facilitate mass retrieving by one or more group of
students of the same piece of course.

\end{document}


